\documentclass[10pt]{article}

\usepackage[T2A]{fontenc}
\usepackage[utf8]{inputenc}
\usepackage[english,russian]{babel}
\usepackage[a4paper,includeheadfoot,top=5mm,bottom=10mm,left=10mm,right=10mm]{geometry}

\usepackage{etoolbox}
\usepackage{lastpage}
\usepackage{fancyhdr}
\usepackage{enumerate}
\usepackage[bookmarks]{hyperref}
\usepackage{framed}

\usepackage{xparse}
\usepackage{amsmath}
\usepackage{amssymb}
\usepackage{amsthm}
\usepackage{mathtools}
\usepackage{cancel}
\usepackage{tikz}

\theoremstyle{plain}
\newtheorem{lemma}{Лемма}
\newtheorem*{lemma*}{Лемма}

\theoremstyle{remark}
\newtheorem*{remark}{Замечание}

\newcommand{\R}{\mathbb{R}}
\renewcommand{\C}{\mathbb{C}}
\newcommand{\N}{\mathbb{N}}
\newcommand{\Q}{\mathbb{Q}}
\newcommand{\Z}{\mathbb{Z}}
\newcommand{\E}{\mathbb{E}}
\newcommand{\D}{\mathbb{D}}
\renewcommand{\P}{\mathbb{P}}

\makeatother

\renewcommand{\le}{\leqslant}
\renewcommand{\ge}{\geqslant}
\renewcommand{\leq}{\leqslant}
\renewcommand{\geq}{\geqslant}

\DeclareMathOperator*{\res}{res}

\frenchspacing

\pagestyle{fancy}

\lhead{Adjoint-State Method for Fast
Demographic History Inference, весна 2021/22}
\chead{}
\rhead{Иван Климов}

\lfoot{}
\cfoot{\thepage\t{/}\pageref*{LastPage}}
\rfoot{}

\title{Отчет?}
\author{Климов Иван}
\date{}
\begin{document}
\maketitle

В дальнейших рассуждениях будем считать число популяций равное единице.

Функция, которую мы хотим оптимизировать:
\begin{equation*}
  h(\Theta | S) = \prod_{d = 0 \dotsc n} \frac{e^{-M[d]} M[d]^{S[d]}}{S[d]!}
\end{equation*}

Посмотрим что вообще хочется знать в общем случае:
\begin{equation*}
  \frac{\mathrm{d}h}{\mathrm{d}\theta} = \frac{\partial h}{\partial \phi} \frac{\partial \phi}{\partial \theta} + \frac{\partial h}{\partial \theta}
\end{equation*}

Заметим, что в $h$ не используется $\theta$ в явном виде. Поэтому $\frac{\partial h}{\partial \theta} = 0$. 
Попробуем избежать подсчета $\frac{\partial \phi}{\partial \theta}$ при помощи Adjoint-State Method. Пусть:
\begin{equation*}
  \mathcal{L}(\phi(\theta), \theta) = h(\phi(\theta), \theta) + \lambda \cdot F(\phi(\theta), \theta)
\end{equation*}
Что такое $F$? Разумеется условие на то, что $\phi(\theta)$ является ровно корнем нашего дифференциального уравнения.
Продифференцируем Лагранжиан:
\begin{equation*}
  \frac{\mathrm{d}\mathcal{L}}{\mathrm{d} \theta} = 
  \frac{\partial h}{\partial \phi} \frac{\partial \phi}{\partial \theta} +
  \lambda \cdot \left(\frac{\partial F}{\partial \phi} \frac{\partial \phi}{\partial \theta} + \frac{\partial F}{\partial \theta}\right) =
  \left(\frac{\partial h}{\partial \phi} + \lambda \frac{\partial F}{\partial \phi}\right) \cdot \frac{\partial \phi}{\partial \theta} + 
  \lambda \frac{\partial F}{\partial \theta}
\end{equation*}
Распишем размерности функций чтобы понимать что вообще происходит(в одномерном случае):
\begin{itemize}
  \item $h$ выдает в качестве результата конкретный likehood, поэтому она действует в $\R$
  \item $\theta$ можно воспринимать как "константную" функцию, она всегда выдает $\R^M$, где $M$ --- количество параметров.
  \item $\phi$ выдает в качестве результата все ответы на константной сетке, то есть вектор $\R^{N^P}$ или $\R^N$ в случае одной популяции, где $N$ --- количество точек на сетке.
  \item $F$ выдает то же самое что и $\phi$, поэтому результат это вектор $\R^{N^P}$ или $\R^N$ в случае одной популяции.
\end{itemize}
Тогда можем сделать следующие выводы по размерностям производных:
\begin{itemize}
  \item $\frac{\mathrm{d} h}{\mathrm{d} \theta}$ размерность $\R^{1 \times M}$
  \item $\frac{\partial h}{\partial \phi}$ размерность $\R^{1 \times N}$
  \item $\frac{\partial F}{\partial \phi}$ размерность $\R^{N \times N}$
  \item $\frac{\partial \phi}{\partial \theta}$ размерность $\R^{N \times M}$
  \item $\frac{\partial F}{\partial \theta}$ размерность $\R^{N \times M}$
\end{itemize}

Легко видеть что тогда для получения корректного результата $\lambda$ должна иметь размерность $\R^{1 \times N}$.
Давайте выберем $\lambda$, так чтобы множитель занулился и нам не пришлось бы считать $\frac{\partial \phi}{\partial \theta}$. Как именно его выбрать?
Пусть $A \coloneqq \frac{\partial F}{\partial \phi}, b = \frac{\partial h}{\partial \phi}$. Тогда:
\begin{equation*}
  \lambda A = -b \implies A^T \lambda^T = -b^T
\end{equation*}
Таким образом наш вектор $\lambda$ это просто решение понятной системы уравнений. Теперь нам нужно посчитать три производные:

\begin{itemize}
  \item $\boxed{\frac{\partial h}{\partial \phi}}$

  Считать градиент от $n + 1$  произведения не очень хочется, так как производная произведения это не
  произведение производных и там все очень страшно разрастается.

  Тогда можем заметить, что у функции и логарифма одинаковые промежутки монотонности, следовательно можно найти градиент логарифма и это будет то что нам нужно.
  Не забываем о том, что весь аллель-частотный спектр нормируется. Обозначим: $\mathcal{A} = \sum_{i = 0}^N S[i],\; \mathcal{B} = \sum_{i = 0}^N M[i]$. Тогда:
  \begin{gather*}
    \frac{\mathrm{d} \log h}{\mathrm{d} \phi} = \sum_{i = 0}^N -\mathcal{A} \cdot \frac{\mathrm{d}}{\mathrm{d}\phi} \frac{M[i]}{\mathcal{B}} + 
    S[i] \cdot \frac{\mathrm{d} \log M[i]}{\mathrm{d} \phi} - S[i] \cdot \frac{\mathrm{d}\mathcal{B}}{\mathrm{d}\phi} = \\ =
    \sum_{i = 0}^N -\frac{\mathcal{A}}{\mathcal{B}^2} \left(\mathcal{B} \frac{\mathrm{d} M[i]}{\mathrm{d} \phi} - M[i] \frac{\mathrm{d} \mathcal{B}}{\mathrm{d}\phi} \right) +
    \frac{S[i]}{M[i]} \frac{\mathrm{d}M[i]}{\mathrm{d}\phi} - S[i] \frac{\mathrm{d} \mathcal{B}}{\mathrm{d}\phi}
    % \frac{\mathrm{d} \log h}{\mathrm{d} \phi} = \sum_{i = 0}^{n} 
    % \frac{\mathrm{d} M[i]}{\mathrm{d} \phi} \cdot \left(\frac{S[i]}{M[i]} - 1\right)
  \end{gather*}
  
  Свели задачу к нахождению $\frac{\mathrm{d} M[i]}{\mathrm{d} \phi}$ при фиксированном  $i$. Следующим шагом проталкиваем градиент внутрь $M$:
  \begin{gather*}
    \frac{\mathrm{d} M[i]}{\mathrm{d} \phi} = 
    \int_0^1
    \binom{n}{i} x^{i} (1 - x)^{n - i}  \; \mathrm{d}x
  \end{gather*}
  
  Тогда значение АЧС в клетке можно вычислить приближенно через метод трапеций:
  \begin{equation*}
    \frac{\mathrm{d} M[i]}{\mathrm{d} \phi} = \binom{n}{i}\sum_{k = 0}^{N - 1} \frac{x_{k + 1} - x_k}{2} \cdot (x_k^{i} (1 - x_k)^{n - i} + x_{k + 1}^{i} (1 - x_{k + 1})^{n - i})
  \end{equation*}
  
  То есть эта часть решения свелась к нахождению какой-то фиксированной константы, равной численному значению такого интеграла.

  \item $\boxed{\frac{\partial F}{\partial \phi}}$
%  Не особо понятно с последней производной, зато можно расписать вторую(возьмем определение $F$ из Supplementary materials):
%  \begin{equation*}
%    F = \frac{1}{2}\sum_\alpha \frac{\partial^2}{\partial^2 x_\alpha} \left(V^{(\alpha)}\phi\right) - 
%    \sum_\alpha \frac{\partial}{\partial x_\alpha} \left(M^{(\alpha)}\phi\right) - \frac{\partial \phi}{\partial \tau}
%  \end{equation*}
%  Тогда, воспользовавшись тем, что $V^{(\alpha)} = x_\alpha(1 - x_\alpha)$ и $M^{(\alpha)} = \sum_\beta M_{\alpha \leftarrow \beta} (x_\beta - x_\alpha)$ получаем:
%  \begin{equation*}
%    \frac{\partial F}{\partial \phi} = \frac{1}{2} \sum_\alpha \frac{\partial^2}{\partial^2 x_\alpha} V^{(\alpha)} -
%    \sum_\alpha \frac{\partial}{\partial x_\alpha} M^{(\alpha)} = -P + \sum_\alpha \sum_\beta M_{\alpha \leftarrow \beta}
%  \end{equation*}
%  Опять получилось что-то невероятное! Таким образом осталось лишь придумать что делать с $\frac{\partial F}{\partial \theta}$.
  Посмотрим на определение $F$ из основной статьи:
  \begin{equation*}
    F = \frac{1}{2} \sum_{i = 1}^{P} \frac{\partial^2}{\partial^2 x_i} \frac{x_i(1 - x_i)}{\nu_i} \phi -
    \sum_{i = 1}^P \frac{\partial}{\partial x_i} \left(\gamma_i x_i(1 - x_i) + \sum_{j = 1}^P M_{i \leftarrow j} (x_j - x_i) \right)\phi -
    \frac{\partial}{\partial \tau } \phi
  \end{equation*}
  Тогда, сразу воспользовавшись тем что популяция у нас пока одна:
  \begin{equation*}
    \frac{\partial F}{\partial \phi} = -\frac{1}{\nu} -\gamma (1 - 2x)
  \end{equation*}

  \item $\boxed{\frac{\partial F}{\partial \theta}}$

  Если рассмотреть численную схему вычисления $\varphi$, то получим следующее тождество:
  \begin{equation*}
    A \varphi^{t + 1} = \varphi^t + C
  \end{equation*}
  Где $C$ --- какая-то константа домноженная на базисный вектор, отвечающая за мутации, $A$ --- наша трехдиагональная матрица перехода, а $\varphi$ --- это 
  вектора, отвечающие за значения функции на сетке в моментах времени $t + 1$ и $t$. Распишем подробнее что же такое $A$ и $C$ в случае одной популяции:
  \begin{itemize}
    \item
    \begin{equation*}
      C = \frac{\Delta t}{x_1} \cdot \frac{\theta}{x_2 - x_0} \cdot 
      \begin{pmatrix*}
        0 \\
        1 \\
        0 \\
        \vdots \\
        0
      \end{pmatrix*}
    \end{equation*}
    \item Для подсчета матрицы $A$ посмотрим на вывод трехдиагональной системы:
    \begin{gather*}
      \frac{\phi^{t + 1}_j - \phi^t_j}{\Delta t} = \frac{1}{\Delta_j}\left(F^{t + 1}_{j + 1/2} - F^{t + 1}_{j - 1/2}\right) \\ 
      \phi_j^{t + 1} - \frac{\Delta t}{\Delta_j} \left(F^{t + 1}_{j + 1/2} - F^{t + 1}_{j - 1/2}\right) = \phi_j^t
    \end{gather*}
    При этом:
    \begin{gather*}
      F^{t + 1}_{j + 1/2} = \frac{1}{x_{j + 1} - x_j} \frac{1}{2 \nu} \left(x_{j + 1}(1 - x_{j + 1}) \phi^{t + 1}_{j + 1} - x_j(1 - x_j) \phi^{t + 1}_j\right) 
      - \frac{\gamma}{2}\frac{(x_j + x_{j + 1})}{2} \left(1 - \frac{x_j + x_{j + 1}}{2}\right) (\phi_j^{t + 1} + \phi_{j + 1}^{t + 1})
      \\
      F^{t + 1}_{j - 1/2} = \frac{1}{x_{j} - x_{j - 1}} \frac{1}{2 \nu} \left(x_{j}(1 - x_{j}) \phi^{t + 1}_{j} - x_{j - 1}(1 - x_{j - 1}) \phi^{t + 1}_{j - 1}\right) 
      - \frac{\gamma}{2}\frac{(x_{j - 1} + x_{j})}{2} \left(1 - \frac{x_{j - 1} + x_{j}}{2}\right) (\phi_{j - 1}^{t + 1} + \phi_{j}^{t + 1})
    \end{gather*}
    Распишем их разницу в удобном формате:
    \begin{align*}
      F_{j + 1/2}^{t + 1} - F_{j - 1/2}^{t + 1} 
      &= \phi_{j - 1}^{t + 1} \left(\frac{x_{j - 1}(1 - x_{j - 1})}{2\nu(x_j - x_{j - 1})} 
      + \frac{\gamma}{2} \frac{x_{j - 1} + x_j}{2}\left(1 - \frac{x_{j - 1} + x_j}{2}\right)\right)
      \\
      &+ \phi_{j}^{t + 1} \left( -\frac{x_j(1 - x_j)}{2\nu}\left(\frac{1}{x_{j + 1} - x_j} + \frac{1}{x_j - x_{j - 1}} \right) \right)
      \\
      &+ \phi_{j}^{t + 1}\left(\frac{\gamma}{2}\frac{x_{j - 1} + x_j}{2}\left(1 - \frac{x_{j - 1} + x_j}{2}\right) 
                             - \frac{\gamma}{2}\frac{x_j + x_{j + 1}}{2}\left(1 - \frac{x_j + x_{j + 1}}{2}\right)\right)
      \\
      &+ \phi_{j + 1}^{t + 1} \left(\frac{x_{j + 1}(1 - x_{j +1})}{2\nu(x_{j + 1} - x_j)} 
      - \frac{\gamma}{2}\frac{x_j + x_{j + 1}}{2} \left(1 - \frac{x_j + x_{j + 1}}{2}\right) \right)
    \end{align*}
    Тогда, домножив это на $-\frac{\Delta t}{\Delta_j}$ и добавив $1$ к коэффициенту при $\phi_j^{t + 1}$ мы получаем следующие коэффициенты в
    трехдиагональной марице:
    \begin{gather*}
      \boxed{a_j} = -\frac{\Delta t}{\Delta_j}\left(\frac{x_{j - 1}(1 - x_{j - 1})}{2\nu(x_j - x_{j - 1})} 
      + \frac{\gamma}{2} \frac{x_{j - 1} + x_j}{2}\left(1 - \frac{x_{j - 1} + x_j}{2}\right)\right) \\
      \boxed{b_j} = 1 + \frac{\Delta t}{\Delta_j} \left( \frac{x_j(1 - x_j)}{2\nu}\frac{\Delta_j}{(x_{j + 1} - x_j)(x_j - x_{j - 1})} 
      -\frac{\gamma}{2}\frac{x_{j - 1} + x_j}{2}\left(1 - \frac{x_{j - 1} + x_j}{2}\right) 
                             + \frac{\gamma}{2}\frac{x_j + x_{j + 1}}{2}\left(1 - \frac{x_j + x_{j + 1}}{2}\right) \right) \\
      \boxed{c_j} = \frac{\Delta t}{\Delta_j} \left(\frac{\gamma}{2}\frac{x_j + x_{j + 1}}{2} \left(1 - \frac{x_j + x_{j + 1}}{2}\right) 
      - \frac{x_{j + 1}(1 - x_{j +1})}{2\nu(x_{j + 1} - x_j)} \right)
    \end{gather*}
    Производные коэффициентов по $\theta$ очевидны. Все формулы совпадают с исходным кодом $\partial a \partial i$.

    В граничных точках получается следующая ситуация: $a_0 = c_G = 0$. При это $c_0$ и $a_G$ совпадают с формулами выше. Осталось посмотреть на код и понять 
    что происходит с коэффициентами $b$:
    \begin{gather*}
      b_0 = 1 + \frac{\Delta t}{\Delta_j} \left(\frac{x_0 (1 - x_0)}{2\nu(x_1 - x_0)} + \frac{\gamma}{2} \frac{x_0 + x_1}{2} \left(1 - \frac{x_0 + x_1}{2}\right) \right) \\
      b_G = 1 + \frac{\Delta t}{\Delta_j} \left(\frac{x_G(1 - x_G)}{2\nu (x_G - x_{G - 1})} - \frac{\gamma}{2} \frac{x_{G - 1} + x_G}{2} \left(1 - \frac{x_{G - 1} + x_G}{2} \right)  \right)
      + \frac{\Delta t}{\Delta_j} \left(\frac{1}{\nu} + \gamma \cdot x_G \cdot (1 - x_G) \right)
    \end{gather*}
    %Тогда легко видеть, что в равенстве вида $a_j \phi^{t + 1}_{j - 1} + b_j \phi^{t + 1}_j + c_j \phi^{t + 1}_{j + 1} = \phi^t_j$ будут следующие коэффициенты:
    %\begin{gather*}
    %  a_j = - \frac{1}{2}\frac{\Delta t}{\Delta_j} \frac{x_{j - 1}(1 - x_{j - 1})}{x_j - x_{j - 1}} \\
    %  b_j = \frac{x_j(1 - x_j)}{2}\frac{\Delta t}{\Delta_j}\left(\frac{1}{x_{j + 1} - x_j} + \frac{1}{x_j - x_{j - 1}}\right) 
    %  = \frac{x_j(1 - x_j)\Delta t}{(x_{j + 1} - x_j)(x_j - x_{j - 1})} \\
    %  c_j = 1 - \frac{1}{2} \frac{\Delta t}{\Delta_j} \frac{x_{j + 1}(1 - x_{j + 1})}{x_{j + 1} - x_j}
    %\end{gather*}

    Производная будет считаться следующим образом:
    \begin{equation*}
      \frac{\mathrm{d} \phi^{t + 1}}{\mathrm{d} \theta} = \frac{\mathrm{d}A^{-1}}{\mathrm{d}\theta} (\phi^t + C) + A^{-1} \left(\frac{\mathrm{d}\phi^t}{\mathrm{d} \theta} + \frac{\mathrm{d} C}{\mathrm{d} \theta} \right)
    \end{equation*}
  \end{itemize}

\end{itemize}
\end{document}